
\documentclass[fleqn]{article}
\usepackage[utf8]{inputenc}
\usepackage{amsmath}
\usepackage{graphicx}
\usepackage{mathtools}

\oddsidemargin = 2pt
\textwidth = 500pt
\headheight = 5pt
\title{MRB - Iets Uitrekenen}
\author{Vincent van Setten }
\date{April 2021}

\begin{document}

\begin{titlepage}
   \begin{center}
       \vspace*{1cm}
        \Large
       \textbf{MRB}

       \vspace{0.5cm}
        Iets Uitrekenen
            
       \vspace{1.5cm}

       \textbf{Vincent van Setten}

       \vfill
      \includegraphics[width=0.7\textwidth]{hu.png}
       \vfill

       HU University of Applied Sciences Utrecht
            
       \vspace{0.8cm}

       April 2021
            
   \end{center}
\end{titlepage}

\section{Opgave 1 - Breuken Vermenigvuldigen}

\subsection{A}
\begin{align*}
    \frac{7}{2} \cdot \frac{1}{4} = \frac{7}{8}
\end{align*}

\subsection{B}
\begin{align*}
    7 \cdot \frac{1}{2} \cdot \frac{1}{4} = \frac{7}{2} \cdot \frac{1}{4} = \frac{7}{8}
\end{align*}

\subsection{C}
\begin{align*}
    7 \cdot \frac{1}{2 \cdot 4} = \frac{7}{2 \cdot 4} = \frac{7}{8}
\end{align*}

\subsection{D}
\begin{align*}
    - \frac{7}{2} \cdot - \frac{1}{-4} = -\frac{7}{2} \cdot -\frac{1}{4} = -\frac{7}{8}
\end{align*}

\subsection{E}
\begin{align*}
    \frac{7 \cdot c}{-2} \cdot -\frac{1}{-4} = -\frac{7 \cdot c}{-2} \cdot \frac{1}{4} = -\frac{7 \cdot c}{8}
\end{align*}

\clearpage
\section{Opgave 2 - Breuken Vereenvoudigen}
\subsection{A}
\begin{align*}
    \frac{27}{18}=1\frac{9}{18} = 1\frac{1}{2}
\end{align*}

\subsection{B}
\begin{align*}
    -\frac{32}{12}=-2\frac{8}{12}=-2\frac{2}{3}
\end{align*}

\subsection{C}
\begin{align*}
    \frac{c \cdot 56}{-14} = c\cdot \frac{56}{-14} = c \cdot -\frac{56}{14} = c \cdot -4 = -4c
\end{align*}

\subsection{D}
\begin{align*}
    \frac{48}{15 \cdot x} = 3\frac{3}{15 \cdot x} = 3 \frac{1}{5 \cdot x}
\end{align*}

\clearpage
\section{Opgave 3 - Breuken Optellen}
\subsection{A}
\begin{align*}
    \frac{1}{2} + \frac{4}{3} = \frac{1\cdot3}{2\cdot3} + \frac{4\cdot2}{3\cdot2} = \frac{3}{6} + \frac{8}{6} = \frac{11}{6} = 1\frac{5}{6}
\end{align*}

\subsection{B}
\begin{align*}
    \frac{2}{7} + \frac{5}{6} = \frac{2\cdot6}{7\cdot6} + \frac{5\cdot7}{6\cdot7} = \frac{12}{42}+\frac{35}{42} = \frac{47}{42} = 1\frac{5}{42} 
\end{align*}

\subsection{C}
\begin{align*}
    -\frac{2}{-7} + \frac{-5}{6}=\frac{2}{7}+\frac{-5}{6}=\frac{2\cdot6}{7\cdot6}+\frac{-5\cdot7}{6\cdot7}=\frac{12}{42}+\frac{-35}{42}=\frac{-23}{42}
\end{align*}

\subsection{D}
\begin{align*}
    -\frac{2\cdot x}{-7} + \frac{-5}{6}=\frac{-2\cdot x}{-7} + \frac{-5}{6} = \frac{-2x}{-7} + \frac{-5}{6} = \frac{-2x\cdot6}{-7\cdot6} + \frac{-5\cdot-7}{6\cdot-7}= \frac{-12x}{-42} + \frac{35}{-42} = \frac{35-12x}{-42}
\end{align*}

\subsection{E}
\begin{align*}
    -\frac{2 \cdot x}{-7} + \frac{-5}{6\cdot c}= \frac{-2x}{-7}+\frac{-5}{6c}=\frac{-2x\cdot6c}{-7\cdot6c} + \frac{-5\cdot-7}{6c\cdot-7} = \frac{-12cx}{-42c} + \frac{35}{-42c} = \frac{35-12cx}{-42c}
\end{align*}


\clearpage
\section{Opgave 4 - Breuken Opsplitsen}

\subsection{A}
\begin{align*}
    \frac{x+\frac{y}{x}\cdot(2+c)}{(2+c)\cdot x} = \frac{x+\frac{y}{x}}{(2+c)\cdot x} + \frac{2+c}{(2+c)\cdot x}
\end{align*}

\subsection{B}
\begin{align*}
    \frac{(x+5)\cdot(y-2)}{(2+c)\cdot x} = \frac{x+5}{(2 + c) \cdot x} + \frac{y-2}{(2 + c) \cdot x}
\end{align*}

\clearpage 
\section{Opgave 5 - Haakjes Uitvermenigvuldigen}
\subsection{A}
\begin{align*}
    \frac{1}{2} \cdot (\frac{4}{3} + \frac{1}{2}) = \frac{2}{3} + \frac{1}{4} = \frac{8}{12} + \frac{3}{12} = \frac{11}{12}
\end{align*}

\subsection{B}
\begin{align*}
    -\frac{1}{2} \cdot - ( \frac{-4}{3}+\frac{1}{-2}) = \frac{1}{2} \cdot (\frac{-4}{3}+\frac{1}{-2}) = \frac{-2}{3} + \frac{1}{-4} = \frac{8+3}{-12} = \frac{11}{-12}
\end{align*}

\subsection{C}
\begin{align*}
    -(\frac{-4}{3}x + \frac{1}{-2}) = \frac{4}{3}x + \frac{1}{2} 
\end{align*}
     
\subsection{D}
\begin{align*}
    (a-3)\cdot(-2+b) = -2a + ab + 6 - 3b = ab -2a - 3b + 6 
\end{align*}

\subsection{E}
\begin{align*}
    &(2\cdot c - 3) \cdot ( \frac{-4}{3}\cdot x + \frac{1}{-2 \cdot c}) \\ 
    &=(2c - 3) \cdot (\frac{-4x}{3} + \frac{1}{-2c}) \\
    &= (2c \cdot \frac{-4}{3}) + (2c \cdot \frac{1}{-2c}) + (-3 \cdot \frac{-4x}{3}) + (-3 \cdot \frac{1}{-2c}) \\
    &=\frac{8c}{3} + \frac{2c}{-2c} + \frac{-12x}{3} + \frac{-3}{-2c} \\
    &=\frac{2c-3}{-2c} + \frac{8c-12x}{3} \\ 
    &=\frac{6c-9-16c+24cx}{-6c}\\
    &=\frac{24cx - 22c - 9}{-6c} \\ 
    &=\frac{-24cx + 22c + 9}{6c}
\end{align*}

\subsection{F}
\begin{align*}
    (a-3)\cdot(-2 + b -c) = -2a + ab -ac + 6 -3b + c = ab - ac - 2a -3b +c + 6
\end{align*}

\clearpage
\section{Opgave 6 - Buiten Haakjes Halen}
\subsection{A}
\begin{align*}
    \frac{1}{2} \cdot (\frac{4}{3}x + \frac{x}{2}) = \frac{x}{2} \cdot ( \frac{4}{3} + \frac{1}{2} )
\end{align*}

\subsection{B}
\begin{align*}
    (-\frac{4}{3}x^2 + \frac{x^2  \cdot (3 - d)}{2 + c}) = x^2(-\frac{4}{3} + \frac{3-d}{2+c})
\end{align*}

\clearpage
\section{Opgave 7 - Vergelijking Met 1 Onbekende Oplossen}
\subsection{A}
\begin{align*}
(x-3)\cdot (-2 + b) &= 0 \\
-2x + xb +6 -3b &= 0 \\
-2x + xb &= 3b - 6  \\ 
x(b - 2) &= 3b - 6  \\
x &= \frac{3b -6}{b - 2}  \\
x &= \frac{3(b-2)}{b-2}  \\ 
x &= 3 \wedge b \neq 2
\end{align*}

\subsection{B}
\begin{align*}
\frac{-4}{3} \cdot x + \frac{1}{-2 \cdot c} &= c \cdot x - d \\ 
\frac{-4x}{3} + \frac{1}{-2c} &= cx - d \\ 
\frac{-4x}{3} -cx &= \frac{-1}{-2c} - d \\ 
\frac{4x}{3} + cx &= d -\frac{1}{2c} \\ 
x(\frac{4}{3} + c) &= d -\frac{1}{2c} \\ 
x &= \frac{d -\frac{1}{2c}}{c + \frac{4}{3}} \wedge c \neq -\frac{4}{3}
\end{align*}

\subsection{C}
\begin{align*}
5 - (\frac{-4}{3} \cdot x + 2) &= d - \frac{x}{3}\\
5 - (\frac{-4x}{3} + 2) &= d - \frac{x}{3} \\ 
- (\frac{-4x}{3} + 2) + \frac{x}{3} &= d - 5 \\
(\frac{4x}{3} - \frac{2}{2}) + \frac{x}{3} &= d - 5 \\ 
(\frac{8x}{6} + \frac{6}{6}) + \frac{x}{3} &= d - 5 \\ 
\frac{8x+6}{6} + \frac{x}{3} &= d - 5 \\
\frac{24x + 18}{18} + \frac{6x}{18} &= d - 5 \\
\frac{30x + 18}{18} &= d - 5\\ 
30x + 18 &= 18d - 90 \\ 
30x &= 18d - 108 \\ 
x &= \frac{3}{5}d - 3\frac{3}{5} \\ 
x &= \frac{3}{5}(d - 3)
\end{align*}

\subsection{D}
\begin{align*}
\frac{2(x+2)}{-2(x+1)-1}+a &= -b \\ 
\frac{2x+4}{-2x-2-1}+a &= -b \\ 
\frac{2x+4}{-2x-3} &= -a -b \\ 
2x+4 &= (-2x-3)(-a -b) \\ 
2x+4 &= 2ax +2bx +3a +3b \\ 
2x &= 2ax +2bx +3a +3b -4 \\ 
x &= ax +bx + 1\frac{1}{2}a + 1\frac{1}{2}b -2 \\
x -ax -bx &= 1\frac{1}{2}a + 1\frac{1}{2}b - 2\\
x(1  -a -b) &= 1\frac{1}{2}a + 1\frac{1}{2}b -2\\
x &= \frac{1\frac{1}{2}a + 1\frac{1}{2}b - 2}{-a -b+1}\\
\end{align*}

\clearpage
\section{Opgave 7 - Stelsel vergelijkingen oplossen} 
\subsection{A}
\begin{align*}
2x+3y=&-4 \hspace{3pt}\textbf{(1)}\\ 
\frac{y}{2-x}&=2 \hspace{3pt}\textbf{2} \\
\end{align*}

\begin{align*}
y &= 2(2-x) \hspace{3pt}\textbf{2*}\\
y &= 4-2x\\
2x+y &= 4\\
\end{align*}

\begin{align*}
\begin{cases}
 2x + y = 4\\
\underline{2x + 3y= -4}\hspace{5pt}-\\
\end{cases}
\end{align*}
$\begin{aligned}[t]
\hspace{50pt}-2y=8 \\
\hspace{50pt}y = -4
\end{aligned}$

\begin{align*} 
\begin{rcases*}
y = -4\\
2x + 3y = -4\\
\end{rcases*}
\hspace{10pt}\text{Geeft}\hspace{20pt}2x + 3\cdot-4 = -4 \\ 
\hspace{50pt}2x -12 = -4 \\ 
\hspace{50pt}2x = 8
\hspace{50pt}x=4
\end{align*}

\begin{align*}
\text{Dus}\hspace{3pt}x = 4 \wedge y=-4
\end{align*}


\subsection{B}
\textbf{\HUGE Deze klopt niet, maar ik kwam hier niet meer aan toe}
\begin{align*}
2x+3y&= a \hspace{3pt}\textbf{(1)}\\ 
\frac{y}{2-x}&=b \hspace{3pt}\textbf{(2)} \\
\end{align*}

\begin{align*}
2x +3y = -4 \\ 
2x = -3y -4 \\ 
x = \frac{-3}{2}y -2 \hspace{5pt}\textbf{1*}
\end{align*}

\begin{align*}
    \frac{y}{2-\frac{-3}{2}y -2} = b\hspace{5pt}\textbf{2'}\\
\end{align*} 
\begin{align*}
\frac{y}{2-x}=b\\ 
y=b(2-x)\\ 
y = 2b - bx  \hspace{5pt}\textbf{2*}\\ 
\end{align*}




\end{document}
